% !TeX encoding = UTF-8
% !TeX spellcheck = en_US
% !TeX program = lualatex
% !TeX TXS-program:compile = txs:///lualatex/[-shell-escape]

%COMPILE WITH LUALATEX FOR THE FONTS TO WORK!

\documentclass[a3,2col]{commsysposter}
%\documentclass[a3,2col,centertitle,landscape]{commsysposter}

% There are several options that can be passed to the class:
% - a4, a3, a2, a1, a0: Sets the paper size. (default: 3col)
%   Note that often you may want to create an A3 poster and then scale it up to A1 or A0.
% - landscape/portrait: Defines whether the poster uses landscape or portrait orientation (default: portrait).
% - ?col: Sets the number of columns to ? in the root of the poster. (default: 3 columns)
% - centertitle: Centers the poster box titles horizontally.
% - showframes: Displays the bounding boxes of all columns and poster areas. Useful for creating advanced layouts.

% This is needed for LuaLaTeX to convert eps to pdf.
\usepackage{epstopdf}

% For old, non-UFT8, compilers.
\usepackage[utf8]{inputenc}

% These steps enable the LiU fonts.
\usepackage[no-math]{fontspec}
\setmainfont[Path=templateresources/,Ligatures=TeX,BoldFont=*b,UprightFont=*,ItalicFont=*i,BoldItalicFont=*z,SmallCapsFeatures={Letters=SmallCaps}]{georgia} % possible fonts: texgyretermes, texgyrepagella, texgyreschola, texgyrebonum, lmroman10, xits
\setsansfont[Path=templateresources/,Ligatures=TeX,BoldFont=*bold,SmallCapsFeatures={Letters=SmallCaps}]{KorolevLiU} % possible fonts: texgyreheros, texgyreadventor
\setmonofont[Ligatures=TeX,Extension=.otf,BoldFont=*-bold,UprightFont=*-regular,ItalicFont=*-italic,BoldItalicFont=*-bolditalic,SmallCapsFeatures={Letters=SmallCaps}]{texgyrecursor} % möjliga typsnitt: texgyrecursor


% This template uses the geometry package to set the dimensions of the poster.
% If you wish to change the margins, you can use:
% \geometry{top=2cm, bottom=2cm, right=2cm, left=2cm}
% If necessary, the page size can also be adjusted before \begin{document}.

% The template exposes a few additional dimensions to change the appearance of the poster.
% For example, if there are many authors or a long title, the height of the header can be adjusted:
% \addtolength{\posterheaderheight}{10mm}

% The template defines a fixed distance between columns, which can be adjusted:
\setlength{\postercolumnseparation}{2mm}

% The distance between the header and the start of the content can be adjusted here:
\setlength{\postertitletocontentmargin}{1mm}

% The boxes are created using the mdframed package.
% If you want to inject extra arguments to style them differently,
% you can redefine the style "posterstyle" as follows:
\mdfdefinestyle{posterstyle}{
	% Enter your key-value arguments here.
}


% The template defines colors based on the LiU color scheme.
% Change "poster" to "LiU" to get the original color palette.
% Available colors are:
% - postercyan
% - posteryellow
% - posterblue
% - posterpurple
% - posterred
% - postergrey
% - postergreen
% - posterorange.


\usepackage{tikz}
% This template works great with TikZ and externalization.
% There is a known bug where the box background disappears behind an
% externalized element. If externalization is disabled, the background
% reappears.
\usetikzlibrary{external}


% Including PGF for the demo figure to work.
\usepackage{pgfplots}
\pgfplotsset{compat=1.18}
\usepgfplotslibrary{external}

% Enable TikZ externalization.
% Uncomment the following line to activate it:
% \tikzexternalize

% English hyphenation settings
\usepackage[english]{babel}

% For generating dummy text
\usepackage{lipsum}

\title{Title of the Poster}
\author{Authors of the Poster}

% Modify the default affiliation if necessary.
% Uncomment and adjust the line below:
% \affiliation{Division, Department, University, etc.}

% The scale of the logos adjusted using these factors.
% \renewcommand{\leftlogoscalingfactor}{1}
% \renewcommand{\rightlogoscalingfactor}{1}

% To use this template, define the number of columns,
% and in these columns, place the content inside boxes.

\begin{document}
% ==================================================================
% ==================================================================
% ======================= Column I =================================
% ==================================================================
% ==================================================================
% Columns are created by the postercolumn environment.
% The number of columns is defined in the documentclass.
\begin{postercolumn}
	% Boxes are created with the posterbox environment:
	% Syntax:
	% \begin{posterbox}[color of the box]{title}
	%    content...
	% \end{posterbox}
	\begin{posterbox}[posterblue]{XX}
		\lipsum[1]
	\end{posterbox}
	\begin{posterbox}[posteryellow]{Boxbox}
		Test
	\end{posterbox}
	\begin{posterbox}[posterorange]{This is a very boring box, but this shows that the title can be arbitrarily long and will wrap to the next line}
		Nothing to see here.
	\end{posterbox}
	\begin{posterbox}{XXX}
		\lipsum[2-3]
	\end{posterbox}
	\wasplogobox
\end{postercolumn}
% IMPORTANT: To ensure proper functionality, there should not be a paragraph break here (i.e., no two new lines).
% You can have commented content here since "%" suppresses the blank line.
%
% ==================================================================
% ==================================================================
% ======================= Column II ================================
% ==================================================================
% ==================================================================
\begin{postercolumn}
	\begin{posterbox}[postergreen]{Boxbox}
		Test
	\end{posterbox}
	\begin{posterbox}{}
		This has no title!
	\end{posterbox}
	\begin{posterbox}[posterpurple]{Important}
		\lipsum[3-4]
	\end{posterbox}
	\begin{posterbox}{A TikZ figure}
		This figure can be externalized with any\footnote{The color of the background will not be shown. However, when externalization is disabled, the color returns.}

		\begin{tikzpicture}
			\begin{axis}[
				axis y line=center,
				axis x line=middle,
				xmax=10,xmin=-1,
				ymin=-5,ymax=10,
				xlabel=$x$, ylabel=$y$,
				xtick={-10,...,10},
				ytick={-10,...,10},
				width=\textwidth,
				anchor=center,
				]
				\addplot {x^2 - 7*x + 10};
			\end{axis}
		\end{tikzpicture}
	\end{posterbox}
	\begin{posterbox}{References}
		Test
	\end{posterbox}
\end{postercolumn}
\end{document}
